\section{Vorwort}
Das Programm KFlog KFlog ist eine (nicht ganz neue), jedoch noch recht unbekannte Software
f�r Segelflieger. F�r wen KFlog gedacht ist, und was damit gemacht werden kann soll hiermit kurz vorgestellt werden.
KFlog (sprich: Kaa �f Log :-) steht f�r KDE Flight LOGger. KDE ist eine
Benutzeroberfl�che f�r diverse Unix-Derivate, darunter auch Linux und Solaris, f�r welche KFlog verf�gbar ist.
\\
Damit wird der Hauptunterschied zu anderen Programmen wie z.B. StrePla ersichtlich? KFlog ist eine Anwendung, f�r alle Plattformen, auf denen die KDE-Umgebung l�uft, wie z.B. Linux oder Solaris. KFlog steht unter der GPL (Open Source) und ist somit frei verf�gbar.
\\
Die Entwicklung des Programms erfolgt durch ein engagiertes Team, das �ber den ganzen Globus verteilt ist und via Internet miteinander in Verbindung steht.
\\
Die Wurzel von KFlog reicht bis in das Jahr 1998 zur�ck.
Das Projekt wurde von einem Geographie Studenten und einem Segelflieger ins Leben gerufenen und war urspr�nglich dazu gedacht ein bisschen programmieren zu lernen ...
